\section{Future work}

The next step is to execute a full-scale evaluation of our approach using 
more varied phenomena and naturally occurring sentences. 
However, the memory requirements of Alchemy are a limitation that prevents us
from currently executing larger and more complex examples.  The problem arises
because Alchemy considers every possible grounding of every atom, even when a
more focused subset of atoms and inference rules would suffice. There is on-going
work to modify Alchemy so that only the required groundings are incorporated
into the network, reducing the size of the model and thus making it possible to
handle more complex inferences.  We will be able to begin using this new version
of Alchemy very soon and our task will provide an excellent test case for the
modification.

Since Alchemy outputs a probability of entailment, it is necessary to fix a
threshold that separates entailment from nonentailment.
We plan to use machine learning techniques to compute an appropriate threshold
automatically from a calibration dataset such as a corpus of valid and invalid
paraphrases.


\section{Conclusion}

In this paper, we have introduced a system that implements a first step
towards integrating logical semantic representations with
probabilistic weights using methods from Statistical Relational AI,
particularly Markov Logic. We have focused on three phenomena and their
interaction: implicatives, coreference, and word meaning. Taking
implicatives and coreference as categorial and word meaning as
probabilistic, we have used a distributional model to generate
paraphrase appropriateness ratings, which we then transformed into
weights on first order formulas.
The resulting MLN approach is able to correctly solve a number of difficult
textual entailment problems that require handling complex combinations of these
important semantic phenomena.

% The framework we have developed takes a pair of natural
% language sentences as input and parses them into DRS \citep{kamp:book93}
% representations.  It then augments those representations by linking the
% predicates back to the original words in the sentences and incorporating
% coreference information from OntoNotes \citep{hovy:naacl2006}.
% Since DRSs are hierarchical structures, our approach flattens them to a simple
% list of atoms while keeping track of the original structure through DRS labels
% as arguments, allowing inferences to be performed in 
% an MLN. In order to maintain the DRS semantics in the flat logical form, we have 
% hand-written a collection of inference rules that are used in the inference. 
% We also generate a list of inference rules to address the particular
% linguistic phenomena that we are handling.  Categorial rules based
% on implication signatures \citep{nairn:icos2006} are used to handle
% implicativity and factivity.  Weighted rules are used to address issues of
% word meaning in context.


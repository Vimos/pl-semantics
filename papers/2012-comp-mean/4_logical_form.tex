\section{Transforming natural language text to logical form}

In transforming natural language text to logical form, we build on the software
package Boxer \citep{bos:coling2004}. Boxer
is an extension to the C\&C parser \citep{clark:acl04} that transforms a parsed
discourse of one or more sentences into a semantic representation.  Boxer
outputs the meaning of each discourse as a Discourse Representation Structure
(DRS) that closely resembles the structures described by \citet{kamp:book93}.

We chose to use Boxer for two main reasons.  First, Boxer is a wide-coverage
system that can deal with arbitrary text.
% that is able to return a reasonable logical representation of most English
% sentences.  Since our goal is to work with actual texts, it is critical that
% we have a wide-coverage semantic parser.  If we did not, then we would be
% unable to deal with any but the simplest texts.
Second, the DRSs that Boxer produces are close to the standard first-order
logical forms that are required for use by the MLN software package Alchemy. 
Our system transforms Boxer output into a format that Alchemy can read and
augments it with additional information.

[TODO: Another example: in section 4, 1st paragraph, where you say "Our system
transforms Boxer output into a format that Alchemy can read and augments it with
additional information", it would be better to say/add "Our system transforms
DRSs into first order logic formulas. A number of problems arise for these
transformations, namely X, Y, and Z. We address them in ways A and B. While A is
an ad hoc solution, B is a contribution ... " Something of the sort.]

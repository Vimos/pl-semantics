\section{Preliminary Evaluation}

As a preliminary evaluation of our system, we constructed the set of 
demonstrative examples included in this paper to test our system's ability
to handle the previously discussed phenomena and their interactions.  We ran
each example with both a standard first-order theorem prover and Alchemy to
ensure that the examples work as expected. Note that since weights are not
possible when running an example in the theorem prover, any rule that would 
receive a non-zero weight in an MLN is simply treated as a ``hard clause'' following
\citet{bos:emnlp2005}.  For the experiments, we generated a vector space from
the entire New York Times portion of the English Gigaword corpus
\citep{graff:gigaword2003}.

The example entailments evaluated were designed to test the interaction between
the logical and weighted phenomena.  For example, in \eqref{ex:ws-imp-1}, ``fail
to'' is a negatively entailing implicative in a positive environment, so according 
to the theorem prover, $p$ entails both {\it h1} and {\it h2}.  However, using our
weighted approach, Alchemy outputs that {\it h1} is more probable than {\it h2}. 
\begin{covex}\label{ex:ws-imp-1}
\begin{itemize}
  \item[{\it p:}]~~    The U.S. is watching closely as South Korea fails to honor
  U.S. patents\footnote{Sentence adapted from Penn Treebank document wsj\_0020.}
  \item[{\it h1:}]~~~~South Korea does not {\bf observe} U.S. patents
  \item[{\it h2*:}]~~~~South Korea does not {\bf reward} U.S. patents
\end{itemize}
\end{covex}
The first-order approach, which contains inference rules for both
paraphrases as hard clauses, cannot distinguish between
good and bad paraphrases, and considers both of them equally valid. In
contrast, the weighted approach can judge the degree of fit of the two potential
paraphrases. Also, it does so in a context-specific manner, choosing
the paraphrase {\it observe} over {\it reward} in the context of
{\it patents}. 

Our ability to perform a full-scale evaluation is currently limited by problems
in the Alchemy software required to perform probabilistic inference.  This is
discussed more in Section \ref{sec:future}.

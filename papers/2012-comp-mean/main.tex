% RECOMMENDED %%%%%%%%%%%%%%%%%%%%%%%%%%%%%%%%%%%%%%%%%%%%%%%%%%%
\documentclass[graybox]{styles/svmult}

% choose options for [] as required from the list
% in the Reference Guide

\usepackage{mathptmx}       % selects Times Roman as basic font
\usepackage{helvet}         % selects Helvetica as sans-serif font
\usepackage{courier}        % selects Courier as typewriter font
\usepackage{type1cm}        % activate if the above 3 fonts are
                            % not available on your system
%
\usepackage{makeidx}         % allows index generation
\usepackage{graphicx}        % standard LaTeX graphics tool
                             % when including figure files
\usepackage{multicol}        % used for the two-column index
\usepackage[bottom]{footmisc}% places footnotes at page bottom

% see the list of further useful packages
% in the Reference Guide

\makeindex             % used for the subject index
                       % please use the style svind.ist with
                       % your makeindex program





%\usepackage{url}       %%% for including URLs
\usepackage{natbib}
%\usepackage[margin=25mm]{geometry}


\usepackage{wrapfig}
\usepackage{subfig}
\usepackage{amsmath,amssymb}
%\usepackage{styles/covington}


\newcommand{\dhgdrs}[3]
{
    {
    \begin{tabular}{rl}
    \\
	\hspace{-100ex}
    #1 
    \hspace{-3ex}
	\vspace{-5ex}
    \\ & 
    \begin{tabular}{|l|}
    \hline
    ~ \vspace{-2.8ex} \\
    #2
    \\
    ~ \vspace{-3ex} \\
    \hline
    ~ \vspace{-2.5ex} \\
    #3
    \\
    ~ \\    % can't vspace here or the line will come out wrong
    \hline
    \end{tabular}
    \hspace{-3ex}
	\end{tabular}
    }
}
\newcommand{\dhgnegdrs}[2]
{
  \mbox{{\huge $\neg$}\hspace{-3.5ex}\dhgdrs{}{#1}{#2}}
}

\newcommand{\dhgboxed}[1]
{
    {
    \begin{tabular}{|l|}
    \hline
    ~ \vspace{-2ex} \\
    #1
    \\
    ~ \\    % can't vspace here or the line will come out wrong
    \hline
    \end{tabular}
    }
}



\begin{document}

\title*{Integrating Logical Representations with\\ Probabilistic Information
using Markov Logic}
% Use \titlerunning{Short Title} for an abbreviated version of
% your contribution title if the original one is too long
\author{Dan Garrette, Katrin Erk, and Raymond Mooney}
% Use \authorrunning{Short Title} for an abbreviated version of
% your contribution title if the original one is too long
\institute{Dan Garrette \at University of Texas at Austin, \email{dhg@cs.utexas.edu}
\and Katrin Erk \at University of Texas at Austin \email{katrin.erk@mail.utexas.edu} 
\and Raymond Mooney \at University of Texas at Austin \email{mooney@cs.utexas.edu}}
%
% Use the package "url.sty" to avoid
% problems with special characters
% used in your e-mail or web address
%
\maketitle


\abstract*{
  First-order logic provides a powerful and flexible mechanism for
  representing natural language semantics.  However, it is an open
  question of how best to integrate it with uncertain, probabilistic
  knowledge, for example regarding word meaning. This paper describes
  the first steps of an approach to recasting first-order semantics
  into the probabilistic models that are part of Statistical
  Relational AI.  Specifically, we show how Discourse Representation
  Structures can be combined with distributional models for word
  meaning inside a Markov Logic Network and used to successfully
  perform inferences that take advantage of logical concepts such as
  factivity as well as probabilistic information on word meaning in
  context.}

\abstract{
  First-order logic provides a powerful and flexible mechanism for
  representing natural language semantics.  However, it is an open
  question of how best to integrate it with uncertain, probabilistic
  knowledge, for example regarding word meaning. This paper describes
  the first steps of an approach to recasting first-order semantics
  into the probabilistic models that are part of Statistical
  Relational AI.  Specifically, we show how Discourse Representation
  Structures can be combined with distributional models for word
  meaning inside a Markov Logic Network and used to successfully
  perform inferences that take advantage of logical concepts such as
  factivity as well as probabilistic information on word meaning in
  context.}



\section{Introduction}

Logic-based representations of natural language meaning have a long history.
Representing the meaning of language in a first-order logical form is appealing
because it provides a powerful and flexible way to express even complex
propositions. However, systems built solely using first-order logical forms tend
to be very brittle as they have no way of integrating uncertain knowledge.
They, therefore, tend to have high precision at the cost of low recall
\citep{bos:emnlp2005}.

Recent advances in computational linguistics have yielded robust methods that
use statistically-driven probabilistic models.  For example, distributional
models of word meaning have been used successfully to judge paraphrase
appropriateness by representing the meaning of a word in context as a point in a
high-dimensional semantics space \citep{erk:emnlp08,thater:acl2010,erk:acl2010}.
However, these models typically provide only shallow representations of meaning,
handling only individual phenomena instead of providing meaning representations
for complete sentences. It is a long-standing open question how best to
integrate the weighted or probabilistic information coming from such modules
with logic-based representations in a way that allows for reasoning over both. 
See, for example, \citet{hobbs:alj93}.

The goal of this work is to establish a formal system for combining
logic-based meaning representations with probabilistic information into a single
unified framework.  This will allow us to obtain the best of both situations: we
will have the full expressivity of first-order logic and be able to reason with
probabilities.  We believe that this will allow for a more complete and robust
approach to natural language understanding.

While this is a large and complex task, this paper proposes first steps toward
our goal by presenting a mechanism for injecting distributional word-similarity
information from a vector space into a first-order logical form.  
\citet{gardenfors:book2004} uses the interpretation function for this purpose,
such that logical formulas are interpreted over vector space representations.
However, he uses spaces whose dimensions are qualities, like the hue and
saturation of a color or the taste of a fruit. Points in his conceptual spaces
are, therefore, potential entities.  In contrast, the vector spaces that we use
are distributional in nature, and, therefore, cannot be interpreted as potential
entities. A point in such a space is a potential word, defined through
its observed contexts, where the coordinates on each dimension constitute the
co-occurrence count with that respective context item.  For this reason, we
define the link between logical form and vector space through a second mapping
function independent of the interpretation function, which we call the
\emph{lexical mapping} function.

A central property of distributional vector space models is that they
can predict semantic similarity based on proximity in space.  This comes from
the hypothesis that a word's meaning is its use, and similarity in
meaning is reflected in similarity in use (CITE Wittgenstein, Harris,
Firth). When syntax is suitably restrained, vector space models can
also be used to predict substitutability in context
\citep{lin:kdd2001,mitchell:acl2008,erk:emnlp08,thater:acl2010,reisinger:naacl2010,dinu:emnlp2010,vandecruys:emnlp2011}.
So if $A$ and $B$ are words or
expressions that are distributionally similar and that can occur in
the same syntactic contexts, we can substitute $B$ for $A$ in all
contexts that are suitable to both words. This can be described
through the inference rule $A \to B$, with a weight computed from the
vector space. Thus, our main aim in linking logical form to a vector
space in this paper is to project inferences from the vector space to
logical form.

In this paper, we first present our formal framework for projecting inferences
from vector space to logical form.  We then show how that framework can be
applied to a real logical language and vector space to address issues of
ambiguity in word meaning.  Finally, we show how the weighted inference rules
produced by our approach interact appropriately with the first-order logical
form to produce correct inferences.

\section{Background}

\textbf{Textual entailment.}
Recognizing Textual Entailment (RTE) is the task of determining whether one
natural language text, the {\it premise}, implies another, the {\it hypothesis}.
For evaluation of our system, we have chosen to use a variation on RTE in which
we assess the relative probability of entailment for a of set of hypotheses.

We have chosen textual entailment as the mode of evaluation
for our approach because it offers a good framework for testing whether a system
performs correct analyses and thus draws the right inferences from a given text.
As an example, consider \eqref{ex:backgound-rte} below.    
\begin{covex}\label{ex:backgound-rte}
\begin{itemize} \itemsep -3pt
  \item[p:]~~    The wine left a stain.
  \item[h1:]~~~The wine resulted in a stain.
  \item[h2*:]~~~The wine fled a stain.
  \item[h3*:]~~~The wine did not resulted in a stain.
\end{itemize}
\end{covex}
Here, hypothesis {\it h1} is a valid entailment, and should be judged to have
high probability by the system.  Hypothesis {\it h2} should have lower probability since
it uses the wrong sense of {\it leave} and {\it h3} should be low probability
because the logical operator $not$ has reversed the meaning of the premise
statement.

% For example, to test whether a system correctly handles implicative verbs, one
% can use the \emph{premise} $p$ along with the \emph{hypothesis} $h$ in
% \eqref{ex:imp-fact-nested} below. If the system analyses the two sentences
% correctly, it should infer that $h$ holds.
While the most prominent forum using textual entailment is the Recognizing
Textual Entailment (RTE) challenge \citep{dagan:rte2005}, the RTE datasets do
not test the phenomena in which we are interested. For example, in order to
evaluate our system's ability to determine word meaning in context, the RTE pair
would have to specifically test word sense confusion by having a word's context
in the hypothesis be different from the context of the premise.  However, this
simply does not occur in the RTE corpora.  In order to properly test our
phenomena, we construct hand-tailored premises and hypotheses based on
real-world texts.


\textbf{Logic-based semantics.}
Boxer \citep{bos:coling2004} is a software package for wide-coverage semantic
analysis that provides semantic representations in the form of Discourse
Representation Structures \citep{kamp:book93}. It builds on the C\&C CCG parser
\citep{clark:acl04}.

\citet{bos:emnlp2005} describe a system for Recognizing Textual Entailment
(RTE) that uses Boxer to convert both the premise and hypothesis of an RTE pair
into first-order logical semantic representations and then uses a theorem prover
to check for logical entailment. 
% \citet{bos:trec2006} varies this model in order
% to use Boxer in a question answering setting by using Boxer to generate a
% logical representation of a document and a question and attempting to unify the
% two to find an answer to the question.


\noindent\textbf{Distributional models for lexical meaning.} Distributional
models describe the meaning of a word through the context in which it
appears~\citep{landauer97:solution,lund96:producing}, where contexts can be
documents, other words, or snippets of syntactic structure. Distributional
models are able to predict semantic similarity between words based on
distributional similarity and they can be learned in an unsupervised fashion.
Recently distributional models have been used to predict the applicability of
paraphrases in context \citep{erk:emnlp2008,thater:acl2010,reisinger:naacl2010,dinu:emnlp2010,vandecruys:emnlp2011}.
For example, in ``The wine left a stain'', {\it result in} is a better
paraphrase for {\it leave} than {\it flee}, because of the context of {\it wine}
and {\it stains}.  In the sentence ``The suspect left the country'', the
opposite is true: {\it flee} is a better paraphrase. Usually, the distributional
representation for a word mixes all its usages (senses). For the paraphrase
appropriateness task, these representations are then reweighted, extended, or
filtered to focus on contextually appropriate usages.

\noindent\textbf{Markov Logic.} 
In order to perform logical inference with weights, we draw
from the large and active body of work related to Statistical Relational AI
\citep{getoor:book2007}.  Specifically, we make use of Markov Logic Networks
(MLNs) \citep{richardson:mlj06} which employ weighted graphical models to
represent first-order logical formulas. MLNs are appropriate for our approach
because they provide an elegant method of assigning weights to first-order
logical rules, combining a diverse set of inference rules, and performing
probabilistic inference.

An MLN consists of a set of weighted first-order clauses.  It provides a way of
softening first-order logic by making situations in which not all clauses are
satisfied less likely, but not impossible \citep{richardson:mlj06}. More
formally, if $X$ is the set of all propositions describing a world (i.e. the
set of all ground atoms), $\mathcal{F}$ is the set of all clauses in the MLN,
$w_i$ is the weight associated with clause $f_i \in \mathcal{F}$,
$\mathcal{G}_{f_i}$ is the set of all possible groundings of clause $f_i$, and
$\mathcal{Z}$ is the normalization constant, then the probability of a
particular truth assignment $\mathbf{x}$ to the variables in $X$ is defined as:
\[ P(X = \mathbf{x}) = \frac{1}{\mathcal{Z}} \exp\left(\sum_{f_i \in
\mathcal{F}} w_i \sum_{g \in \mathcal{G}_{f_i}}g(\mathbf{x}) \right) =
\frac{1}{\mathcal{Z}} \exp\left(\sum_{f_i \in \mathcal{F}} w_i n_i(\mathbf{x})
\right) \tag{1}\label{e1} \] where $g(\mathbf{x})$ is 1 if $g$ is satisfied and
0 otherwise, and $n_i(\mathbf{x})= \sum_{g\in \mathcal{G}_{f_i}}g(\mathbf{x})$
is the number of groundings of $f_i$ that are satisfied given the current truth
assignment to the variables in $X$. This means that the probability of a truth
assignment rises exponentially with the number of groundings that are satisfied.

Markov Logic has been used previously in other NLP application
(e.g. \citet{poon:emnlp2009}).  However, this paper differs in that it is an
attempt to represent deep logical semantics in an MLN.

While it is possible learn rule weights in an MLN directly from training data,
our approach at this time focuses on incorporating weights computed
by external knowledge sources.  Weights for word meaning rules are computed from
the distributional model of lexical meaning and then injected into the MLN. 
Rules governing implicativity and coreference are given infinite weight
(hard constraints).

We use the open source software package Alchemy \citep{kok:tr05} to perform MLN
inference.

\section{Evaluation and phenomena}

Textual entailment offers a good framework for testing whether a
system performs correct analyses and thus draws the right inferences
from a given text. For example, to test whether a system correctly
handles implicative verbs, one can use the \emph{premise} $p$ along with
the \emph{hypothesis} $h$ in \eqref{ex:imp-fact-nested} below. If the system
analyses the two sentences correctly, it should infer that $h$ holds.
While the most prominent forum using textual entailment is the 
Recognizing Textual Entailment (RTE) challenge \citep{dagan:rte2005},
the RTE datasets 
do not test the phenomena in which we are interested.
For example, in order to evaluate our system's ability to determine word meaning
in context, the RTE pair would have to specifically test word sense confusion by
having a word's context in the hypothesis be different from the context of the
premise.  However, this simply does not occur in the RTE corpora.  In order to
properly test our phenomena, we construct hand-tailored premises and hypotheses
based on real-world texts.


In this paper, we focus on three natural language phenomena and their
interaction: implicativity and factivity, word meaning, and coreference.
The first phenomenon, implicativity and factivity, is concerned with analyzing
the truth conditions of nested propositions.  For example, in the premise of
the entailment pair shown in example 
\eqref{ex:imp-fact-nested}, ``arrange that'' falls under the scope of 
``forget to'' and ``fail'' is under the scope of ``arrange that''.
Correctly recognizing nested propositions is necessary for preventing
false inferences such as the one
in example \eqref{ex:hope-build}.

\begin{example}\label{ex:imp-fact-nested}
\begin{itemize}
  \item[$p$:] Ed did not forget to arrange that Dave 
fail\footnote{Examples \eqref{ex:imp-fact-nested}
and \eqref{ex:imp-fact-hyper} and Figure \ref{fig:imp-sig} are based on examples by
\citet{maccartney:iwcs2009}}
  \item[$h$:] Dave failed
\end{itemize}
\end{example}

\begin{example}\label{ex:hope-build}
\begin{itemize}
  \item[$p$:] The mayor hoped to build a new stadium\footnote{Examples
\eqref{ex:hope-build}, \eqref{ex:prob-wordsense}, \eqref{ex:coref}, and
\eqref{ex:ws-coref} are modified versions of sentences from document wsj\_0126
from the Penn Treebank}
  \item[$h$*:] The mayor built a new stadium
\end{itemize}
\end{example}


For the second phenomenon, word meaning, we address paraphrasing and
hypernymy.  For example, in \eqref{ex:prob-wordsense} ``covering'' is a good
paraphrase for ``sweeping'' while ``brushing'' is not.

\begin{example}\label{ex:prob-wordsense}
\begin{itemize}
  \item[$p$:]   A stadium craze is \textbf{sweeping} the country
  \item[$h_1$:] A stadium craze is \textbf{covering} the country
  \item[$h_2$*:] A stadium craze is \textbf{brushing} the country
\end{itemize}
\end{example}

The third phenomenon is coreference, as illustrated in \eqref{ex:coref}.  For
this example, to correctly judge the hypothesis as entailed, it is necessary
to recognize that ``he'' corefers with ``Christopher'' and ``the new ballpark''
corefers with ``a replacement for Candlestick Park''.

\begin{example}\label{ex:coref}
\begin{itemize}
  \item[$p$:] George Christopher has been a critic of the plan to build a
  replacement for Candlestick Park. As a result, he won't endorse the new ballpark.
  \item[$h$:] Christopher won't endorse a replacement for Candlestick Park.
\end{itemize}
\end{example}

Some natural language phenomena are most naturally treated as categorial, while 
others are more naturally treated using weights or 
probabilities. In this paper, we treat implicativity and coreference
as categorial phenomena, while using a probabilistic approach to word
meaning. 


\section{Transforming natural language text to logical form}

In transforming natural language text to logical form, we build on the software
package Boxer \citep{bos:coling2004}. % Boxer
% is an extension to the C\&C parser \citep{clark:acl04} that transforms a parsed 
% discourse of one or more sentences into a semantic representation.  Boxer
% outputs the meaning of each discourse as a Discourse Representation Structure
% (DRS) that closely resembles the structures described by \citet{kamp:book93}.
%
We chose to use Boxer for two main reasons.  First, Boxer is a
wide-coverage system that can deal with arbitrary text. 
%  that is able to return a reasonable logical representation
% of most English sentences.  Since our goal is to work with actual texts,
% it is critical that we have a wide-coverage semantic parser.  If
% we did not, then we would be unable to deal with any but the simplest texts. 
Second, the DRSs that Boxer produces are close to the standard first-order
logical forms that are required for use by the MLN software package 
Alchemy.  Our system transforms Boxer output into a format that Alchemy can read and 
augments it with additional information.

To demonstrate our transformation procedure, consider again the premise of
example \eqref{ex:imp-fact-nested}.  When given to Boxer, the sentence produces
the output given in Figure \ref{fig:boxer-drs}.  We then transform this output
to the format given in Figure \ref{fig:canonical-drs}.


\begin{small}
\begin{figure}
  \centering
  \subfloat[Output from Boxer]{\label{fig:boxer-drs}
		\dhgdrs{}{{\footnotesize x0~ x1}}{
			{\footnotesize named(x0,ed,per)} \\
			\vspace{-.7ex}
			{\footnotesize named(x1,dave,per)} \\
			\vspace{-.7ex}
			\dhgnegdrs{{\footnotesize x2~ x3}}{
				\vspace{-.7ex}
				{\footnotesize forget(x2)} \\
				\vspace{-.7ex}
				{\footnotesize event(x2)} \\
				\vspace{-.7ex}
				{\footnotesize agent(x2,x0)} \\ 
				\vspace{-.7ex}
				{\footnotesize theme(x2,x3)} \\ 
				\vspace{-.7ex}
				\dhgdrs{{\footnotesize x3:}}{{\footnotesize x4~ x5}}{
					\vspace{-.7ex}
					{\footnotesize arrange(x4)} \\ 
					\vspace{-.7ex}
					{\footnotesize event(x4)} \\ 
					\vspace{-.7ex}
					{\footnotesize agent(x4,x0)} \\
					\vspace{-.7ex}
					{\footnotesize theme(x4,x5)} \\ 
					\vspace{-.7ex}
					\dhgdrs{{\footnotesize x5:}}{{\footnotesize x6}}{ 
						\vspace{-.7ex}
						{\footnotesize fail(x6)} \\ 
						\vspace{-.7ex}
						{\footnotesize event(x6)} \\ 
						\vspace{-.7ex}
						{\footnotesize agent(x6,x1)}
						\vspace{-2ex}
					} 
					\vspace{-1.5ex}
				} 
				\vspace{-1.5ex}
			}
			\vspace{-1.5ex}
		}
	}                
  ~~$\xrightarrow{\text{ transforms to }}$
  \subfloat[Canonical form]{\label{fig:canonical-drs}
  	\dhgboxed{
		\vspace{-.7ex}
		{\footnotesize named(l0, ne\_per\_ed\_d\_s0\_w0, z0)} \\
		\vspace{-.7ex}
		{\footnotesize named(l0, ne\_per\_dave\_d\_s0\_w7, z1)} \\
		\vspace{-.7ex}
		{\footnotesize not(l0, l1)} \\
		\vspace{-.7ex}
		{\footnotesize pred(l1, v\_forget\_d\_s0\_w3, e2)} \\
		\vspace{-.7ex}
		{\footnotesize event(l1, e2)} \\
		\vspace{-.7ex}
		{\footnotesize rel(l1, agent, e2, z0)} \\
		\vspace{-.7ex}
		{\footnotesize rel(l1, theme, e2, l2)} \\
		\vspace{-.7ex}
		{\footnotesize prop(l1, l2)} \\
		\vspace{-.7ex}
		{\footnotesize pred(l2, v\_arrange\_d\_s0\_w5, e4)} \\
		\vspace{-.7ex}
		{\footnotesize event(l2, e4)} \\
		\vspace{-.7ex}
		{\footnotesize rel(l2, agent, e4, z0)} \\
		\vspace{-.7ex}
		{\footnotesize rel(l2, theme, e4, l3)} \\
		\vspace{-.7ex}
		{\footnotesize prop(l2, l3)} \\
		\vspace{-.7ex}
		{\footnotesize pred(l3, v\_fail\_d\_s0\_w8, e6)} \\
		\vspace{-.7ex}
		{\footnotesize event(l3, e6)} \\
		\vspace{-.7ex}
		{\footnotesize rel(l3, agent, e6, z1)}
		\vspace{-2ex}
	}
  }
  \caption{Converting the premise of \eqref{ex:imp-fact-nested} from
  Boxer output to MLN input}
  \label{fig:boxer-conversion}
\end{figure}
\end{small}

\noindent\textbf{Flat structure.}
In Boxer output, nested propositional statements are represented as nested sub-DRS structures.
  For example, in the premise of
(\ref{ex:imp-fact-nested}), the verbs ``forget to'' and ``arrange that'' both
introduce nested propositions, as is shown in Figure \ref{fig:boxer-drs} where
DRS {\it x3} (the ``arranging that'') is the {\it theme} of ``forget to'' and
DRS {\it x5} (the ``failing'') is the {\it theme} of ``arrange that''.  

In order to write logical rules about the truth conditions of nested
propositions, the structure has to be flattened. However, it is
clearly not sufficient to just conjoin all propositions at the top
level. Such an approach, applied to example
\eqref{ex:hope-build}, would yield $(hope(x_1) \land theme(x_1, x_2)
\land build(x_2) \land \ldots)$, leading to the wrong inference that
the stadium was built. 
Instead, we add a new argument to each predicate that names the DRS in
which the predicate originally occurred. Assigning the label
\textit{l1} to the DRS containing the predicate \textit{forget}, we
add {\it l1} as the first argument to the atom
{\it pred(l1, v\_forget\_d\_s0\_w3, e2)}.\footnote{The extension to the word,
such as {\it d\_s0\_w3} for ``forget'', is an index providing the location of
the original word that triggered this atom; this is addressed in more detail
shortly.} Having flattened the structure, we need to re-introduce the
information about relations between DRSs. For this we use predicates {\it not}, {\it imp},
and {\it or} whose arguments are DRS labels. For example, $not(l0, l1)$ states 
that $l1$ is inside $l0$ and negated.  Additionally, an atom $prop(l0, l1)$
indicates that DRS $l0$ has a subordinate DRS labeled $l1$.  

One important consequence of our flat structure is that the truth
conditions of our representation no longer coincide with the truth conditions of
the underlying DRS being represented.  
For example, we do not directly express the fact that the ``forgetting'' is actually
negated, since the negation is only expressed as a relation between DRS
labels. 
To access the information encoded in relations between DRS labels, we
add predicates that
capture the truth conditions of the underlying DRS.  We use the  predicates
$true(label)$ and $false(label)$ that state whether the DRS referenced by
$label$ is {\it true} or {\it false}.  We also add rules that
govern how the predicates for logical operators interact with these truth
values.  For example, the rules in \eqref{ex:neg-op-rules} state that if a DRS
is {\it true}, then any negated subordinate must be {\it false} and vice versa.

\begin{equation}\label{ex:neg-op-rules}
\forall~p~n.[not(p,n) \rightarrow (true(p) \leftrightarrow false(n)) \land
(false(p) \leftrightarrow true(n))]
\end{equation}


\noindent\textbf{Injecting additional information into the logical form.}
We want to augment Boxer output with additional
information, for example gold coreference annotation for sentences
that we subsequently analyze with Boxer. In order to do so, we need to
be able to tie predicates in the Boxer output back to words in the
original sentence. Fortunately, the optional ``Prolog'' output format from Boxer
provides the sentence and word indices from the original sentence.  When
parsing the Boxer output, we extract these indices and concatenate them to the
word lemma to specific the exact occurrence of the lemma that is under
discussion.  For example, the atom {\it pred(l1, v\_forget\_d\_s0\_w3, e2)}
indicates that event {\it e2} refers to the lemma ``forget'' that appears in the
$0^{th}$ sentence of discourse {\it d} at word index 3.


\noindent\textbf{Atomic formulas.}
We represent the words from the sentence as arguments instead of
predicates in order to simplify the set of inference rules we need to
specify. Because our flattened structure requires that the inference
mechanism be reimplemented as a set of logical rules, it is desirable
for us to be able to write general rules that govern the interaction
of atoms. With the representation we have chosen, we can quantify over
all predicates or all relations. For example, the rule in
\eqref{ex:second-order-rule} states that a predicate is accessible if
it is found in an out-scoping DRS.
%  If, instead, we used atoms such as
% {\it forget(l1, e2)} and {\it agent(l1, e2, z0)} then we would be
% unable to write such simple inference rules since this would require
% quantification over predicates, a second-order property unavailable
% for first-order inference.

\begin{equation}\label{ex:second-order-rule}
\forall~l_1~l_2.[outscopes(l_1,l_2) \rightarrow \forall~p~x.[pred(l_1,p,x) \rightarrow pred(l_2,p,x)]]
\end{equation}
 
We use three different predicate symbols to distinguish three types of atomic concepts: predicates, named entities, and
relations.  Predicates and named entities represent words that appear in the
text.  For example, {\it named(l0, ne\_per\_ed\_d\_s0\_w0, z0)} indicates that
variable {\it z0} is a person named ``Ed'' while 
{\it pred(l1, v\_forget\_d\_s0\_w3, e2)} says that {\it e2} is a
``forgetting to'' event.  
Relations capture the relationships between words.  For example, 
{\it rel(l1, agent, e2, z0)} indicates that {\it z0}, ``Ed'', is the ``agent''
of the ``forgetting to'' event {\it e2}.




\section{Handling the phenomena}

\subsection*{Implicatives and factives}

\citet{nairn:icos2006} presented an approach to the treatment of inferences
involving implicatives and factives.  Their approach identifies an ``implication
signature'' for every implicative or factive verb that determines the truth
conditions for the verb's nested proposition, whether in a positive or negative
environment.  Implication signatures take the form ``$x/y$'' where $x$
represents the implicativity in the the positive environment and $y$ represents
the implicativity in the negative environment.  Both $x$ and $y$ have three
possible values: ``+'' for positive entailment, meaning the nested proposition
is entailed, ``-'' for negative entailment, meaning the negation of the proposition
is entailed, and ``o'' for ``null'' entailment, meaning that neither the
proposition nor its negation is entailed. Figure \ref{fig:imp-sig} gives
concrete examples.% for two signatures.

% For example, the verb ``managed to'' has positive entailment in the {\it true}
% case and negative entailment under negation.  So, {\it he managed to escape
% $\vDash$ he escaped} while {\it he did not manage to escape $\vDash$ he did not
% escape}.  On the other hand, the verb ``refused to'' has negative entailment
% in the positive case and ``null'' entailment under negation.  So, {\it he
% refused to fight $\vDash$ he did not fight} but {\it he did not refuse to fight}
% entails neither {\it he fought} nor {\it he did not fight}.

\begin{figure}
\begin{center}
  \begin{tabular}{l c l}
    \hline
   	 & ~~~~~~signature~~~~~~ &  \multicolumn{1}{c}{example} \\
   	\hline
   	managed to       & +/- & he managed to escape $\vDash$ he escaped \\
   	                 &     & he did not manage to escape $\vDash$ he did not escape \\
   	\hline
%    	was forced to    & +/o & he was forced to sell $\vDash$ he sold \\
%    	                 &     & he was not forced to sell $?$ he sold \\
%    	\hline
%    	was permitted to & o/- & he was permitted to leave $?$ he left \\
%    	                 &     & he was not permitted to leave $\vDash$ he did not leave \\
%    	\hline
%    	forgot to        & -/+ & he forgot to pay $\vDash$ he did not pay \\
%    	                 &     & he did not forget to pay $\vDash$ he paid \\
%    	\hline
   	refused to       & -/o & he refused to fight $\vDash$ he did not fight \\
   	                 &     & he did not refuse to fight $\nvDash$ \{he fought, he did not fight\} \\
   	\hline
%    	hesitated to     & o/+ & he hesitated to ask $?$ he asked\\
%    	                 &     & he did not hesitate to ask $\vDash$ he asked \\
%    	\hline
%    	admitted that    & +/+ & he admitted that he knew $\vDash$ he knew \\
%    	                 &     & he did not admit that he knew $\vDash$ he knew \\
%    	\hline
%    	pretended that   & -/- & he pretended he was sick $\vDash$ he was not sick \\
%    	                 &     & he did not pretend he was sick $\vDash$ he was not sick\\
%    	\hline
%    	wanted to        & o/o & he wanted to fly $?$ he flew \\  
%    	                 &     & he did not want to fly $?$ he flew \\
%    	\hline
  \end{tabular}
\end{center}
\caption{Implication Signatures}
\label{fig:imp-sig}
\end{figure}

We use these implication signatures to automatically generate rules that
license specific entailments in the MLN.  Since ``forget to'' has implication
signature ``-/+'', we generate the two rules in 
(\ref{ex:imp-fact-rules-managed}).

\begin{small}
\begin{example}\label{ex:imp-fact-rules-managed}
\begin{itemize}
   \item[(a)] $\forall~l_1~l_2~e.[(pred(l_1,``forget",e) \land true(l_1) \land rel(l_1,``theme",e,l_2) \land prop(l_1,l_2)) \rightarrow false(l_2)]]$\footnote{Occurrence-indexing on the predicate ``forget'' has been left out for brevity.}
   \item[(b)] $\forall~l_1~l_2~e.[(pred(l_1,``forget",e) \land false(l_1) \land rel(l_1,``theme",e,l_2) \land prop(l_1,l_2)) \rightarrow true(l_2)]$
\end{itemize}
\end{example}
\end{small}

To understand these rules, consider (\ref{ex:imp-fact-rules-managed}a).  The rule
says that if the atom for the verb ``forget to'' appears in a DRS that has been
determined to be {\it true}, then the DRS representing any ``theme'' proposition
of that verb should be considered {\it false}.  Likewise,
(\ref{ex:imp-fact-rules-managed}b) says that if the occurrence of ``forget to''
appears in a DRS determined to be {\it false}, then the theme DRS should be
considered {\it true}.

Note that when the implication signature indicates a ``null'' entailment, no
rule is generated for that case.  This prevents the MLN from licensing
entailments related directly to the nested proposition, but still allows for
entailments that include the factive verb.  So {\it he wanted to fly} entails
neither {\it he flew} nor {\it he did not fly}, but it does still license {\it
he wanted to fly}.  

% \textbf{KE: The next paragraphs can be cut or shortened severely if needs be.} 
% This approach allows us to capture the consequences of the implication signature
% simply and cleanly with first-order logical rules.  The rules are also generic
% enough to work under various levels of nesting and negation.  Consider again
% (\ref{ex:imp-fact-nested}), which involves nested proposition-introducing verbs
% under negation.  The premise DRS and its flat
% representation are given in Figure \ref{fig:boxer-conversion}.
% The main premise DRS, {\it l0} always {\it true} and the negation modifier ``not''
% entails that the sub-DRS {\it l1} is {\it false}.  Since the verb ``forget to'' is
% positively entailing under negation, the nested DRS {\it l2} is judged as
% {\it true}.  Finally, since the verb ``arrange'' is positively entailing in the
% positive environment, its nested DRS {\it l3} is judged {\it true}. Since the
% ``failing'' act referenced in the hypothesis occurs in a {\it true} DRS, the
% entailment holds.



\subsection*{Ambiguity in word meaning}

In order for our system to be able to make correct natural language inference,
it must be able to handle paraphrasing and deal with hypernymy.  For example,
in order to license the entailment pair in (\ref{ex:syn-hyp-pos}), the system must
recognize that ``owns'' is a valid paraphrase for ``has'', and that ``car'' is a hypernym
of ``convertible''.

\begin{example}\label{ex:syn-hyp-pos}
{\it p:} Ed has a convertible \\
{\it h:} Ed owns a car
\end{example}

In this section we discuss our probabilistic approach to paraphrasing.
In the next section we discuss how this approach is extended to cover
hypernymy. A central problem to solve in the context of paraphrases is
that they are context-dependent. Consider again example
\eqref{ex:prob-wordsense} and its two hypotheses.  The first
hypothesis replaces the word ``sweeping'' with a paraphrase that is
valid in the given context, while the second uses an incorrect
paraphrase. 

To incorporate paraphrasing information into our system, we first
generate rules stating all paraphrase relationships that may
\emph{possibly} apply to a given predicate/hypothesis pair, using
WordNet  \citep{miller:wordnet2009} as a resource. Then we
associate those rules with weights to signal contextual adequacy. For
any two occurrence-indexed words $w_1, w_2$ occurring anywhere in the
premise or hypothesis, we check
whether they co-occur in a WordNet synset. If $w_1, w_2$ have a common synset, 
we generate rules of the form $\forall~l~x.[pred(l,w_1,x) \leftrightarrow
pred(l,w_2,x)]$ to connect them. For named entities, we perform a similar routine:
for each pair of matching named entities found in the premise and hypothesis, we
generate a rule  $\forall~l~x.[named(l,w_1,x) \leftrightarrow named(l,w_2,x)]$.

We then use the distributional model of \citet{erk:acl2010} to compute
paraphrase appropriateness. In the case of \eqref{ex:prob-wordsense}
this means measuring the cosine similarity between the vectors for
``sweep'' and ``cover'' (and between ``sweep'' and ``brush'') in the
sentential context of the premise. MLN formula weights are expected
to be log-odds (i.e., $\log (P/(1-P))$ for some probability $P$), so we
rank all possible paraphrases of a given word $w$ by their cosine similarity
to $w$ and then give them probabilities that decrease by rank according to a
Zipfian distribution.  So, the $k^{th}$ closest paraphrase by cosine similarity
will have probability $P_k$ given by \eqref{eq:zipf}:

\begin{equation}\label{eq:zipf}
P_k \sim 1/k
\end{equation}

% \begin{wrapfigure}{r}{.2\textheight}
% \begin{center}
% \begin{tabular}{lll}
% \hline
%       & P(p) & W(p)   \\
% \hline
% cover &  0.069 & -2.602  \\
% brush &  0.021 & -3.842 \\ 
% \hline
% \end{tabular}
% \end{center}
% \caption{{\small Paraphrase probabilities (left) and log-odds weights (right)}}
% \label{fig:para-weights}
% \end{wrapfigure}

The generated rules are given in \eqref{ex:paraphrase-rules} with the actual
weights calculated for example \eqref{ex:prob-wordsense}.  Note that the
valid paraphrase ``cover'' is given a higher weight than the incorrect 
paraphrase ``brush'', which allows the MLN inference procedure to judge $h_1$ as
a more likely entailment than $h_2$.\footnote{
Because weights are calculated according to the equation $\log(P/(1-P))$, any
paraphrase that has a probability of less than 0.5 will have a negative weight. 
Since most paraphrases will have probabilities less than 0.5, most will yield
negative rule weights.  However, the inferences are still handled properly in
the MLN because the inference is dependent on the {\em relative} weights.}
This same result would not be achieved if we did not take context into
consideration because, without context, ``brush'' is a more likely paraphrase of
``sweep'' than ``cover''.

\begin{example}\label{ex:paraphrase-rules}
\begin{itemize}
  \item[(a)] -2.602~ $\forall~l~x.[pred(l,``v\_sweep\_p\_s0\_w4",x) \leftrightarrow pred(l,``v\_cover\_h\_s0\_w4",x)]$
  \item[(b)] -3.842~ $\forall~l~x.[pred(l,``v\_sweep\_p\_s0\_w4",x) \leftrightarrow pred(l,``v\_brush\_h\_s0\_w4",x)]$
\end{itemize}
\end{example}

Since Alchemy outputs a probability of entailment and not a binary judgment, it
is necessary to specify a probability threshold indicating entailment.  
An appropriate threshold between "entailment" and "non-entailment" will be one
that separates the probability of an inference with the valid rule from the
probability of an inference with the invalid rule.  
While we plan to automatically induce a threshold in the future, our current
implementation uses a value set manually.


\subsection*{Hypernymy}

Like paraphrasehood, hypernymy is context-dependent: In ``A bat flew
out of the cave'', ``animal'' is an appropriate hypernym for ``bat'',
but ``artifact'' is not. So we again use distributional similarity to
determine contextual appropriateness. However, we do not directly
compute cosine similarities between a word and its potential hypernym.
We can hardly assume ``baseball bat'' and ``artifact'' to occur in
similar distributional contexts. So instead of checking for similarity
of ``bat'' and ``artifact'' in a given context, we check ``bat'' and
``club''. That is, we pick a synonym or close hypernym of the word in
question (``bat'') that is also a WordNet hyponym of the hypernym to
check (``artifact'').

A second problem to take into account is the interaction of hypernymy
and polarity. While  \eqref{ex:syn-hyp-pos} is a valid pair, 
\eqref{ex:syn-hyp-neg} is not, because ``have a convertible'' is under
negation. So, we create weighted rules of the form $hypernym(w, h)$,
along with inference rules to guide their interaction with polarity.
We create these rules for all pairs of words $w, h$ in premise and
hypothesis such that $h$ is a hypernym of $w$, again using WordNet to
determine potential hypernymy. 

\begin{example}\label{ex:syn-hyp-neg}
{\it p:} Ed does not have a convertible \\
{\it h:} Ed does not own a car
\end{example}
 
Our inference rules governing the interaction of hypernymy and
polarity are given in (\ref{ex:hypernym-rules}).
The rule in (\ref{ex:hypernym-rules}a) states that in a positive environment,
the hyponym entails the hypernym while the rule in (\ref{ex:hypernym-rules}b)
states that in a negative environment, the opposite is true: the hypernym
entails the hyponym.

\begin{example}\label{ex:hypernym-rules}
\begin{itemize}
  \item[(a)] $\forall~l~p_1~p_2~x.[(hypernym(p_1,p_2) \land true(l) \land pred(l,p_1,x)) \rightarrow pred(l,p_2,x)]]$
  \item[(b)] $\forall~l~p_1~p_2~x.[(hypernym(p_1,p_2) \land false(l) \land pred(l,p_2,x)) \rightarrow pred(l,p_1,x)]]$
\end{itemize}
\end{example}



\subsection*{Making use of coreference information}

As a test case for incorporating additional resources into Boxer's logical form,
we used the coreference data in OntoNotes \citep{hovy:naacl2006}.  However,
the same mechanism would allow us to transfer information into Boxer output from
arbitrary additional NLP tools such as 
automatic coreference analysis tools or semantic role labelers. 
Our system uses coreference information into two distinct ways.

The first way we make use of coreference data is to copy atoms describing a
particular variable to those variables that corefer. 
% Making use of coreference information is often essential in natural language
% inference.  Certain coreferring expressions, however, are more complex. 
Consider again example (\ref{ex:coref}) which has a two-sentence premise. 
This inference requires recognizing that the ``he'' in the second sentence of
the premise refers to ``George Christopher'' from the first sentence.  Boxer
alone is unable to make this connection, but if we receive this information as
input, either from gold-labeled data or a third-party coreference tool, we are
able to incorporate it.  Since Boxer is able to identify the index of the word
that generated a particular predicate, we can tie each predicate to any related
coreference chains.  Then, for each atom on the chain, we can inject copies
of all of the coreferring atoms, replacing the variables to match.
For example, the word ``he'' generates an atom  
{\it pred(l0, male, z5)}\footnote{Atoms simplified for brevity} and
``Christopher'' generates atom {\it named(l0, christopher, x0)}. So, we can
create a new atom by taking the atom for ``christopher'' and replacing the 
label and variable with that of the atom for ``he'',
generating  {\it named(l0, christopher, x5)}.

As a more complex example, the coreference information will inform us that
``the new ballpark'' corefers with ``a replacement for Candlestick Park''.
However, our system is currently unable to handle this coreference correctly at
this time because, unlike the previous example, the expression ``a replacement
for Candlestick Park'' results in a complex three-atom conjunct with two 
separate variables: {\it pred(l2, replacement, x6)}, {\it rel(l2, for, x6, x7)}, and 
{\it named(l2, candlestick\_park, x7)}.  Now, unifying with the atom for ``a
ballpark'', {\it pred(l0, ballpark, x3)}, is not as simple as replacing the
variable because there are two variables to choose from.  Note that it would 
{\it not} be correct to replace
both variables since this would result in a unification of ``ballpark'' with
``candlestick\_park'' which is wrong.  Instead we must determine that {\it x6}
should be the one to unify with {\it x3} while {\it x7} is replaced with a fresh
variable.  The way that we can accomplish this is to look at the dependency
parse of the sentence that is produced by the C\&C parser is a precursor to running
Boxer.  By looking up both ``replacement'' and ``Candlestick Park'' in the
parse, we can determine that ``replacement'' is the head of the phrase, and thus
is the atom whose variable should be unified.  So, we would create new atoms,
{\it pred(l0, replacement, x3)}, {\it rel(l0, for, x3, z0)}, and 
{\it named(l0, candlestick\_park, z0)}, where {\it z0} is a fresh variable.


The second way that we make use of coreference information is to extend the
sentential contexts used for calculating the appropriateness of paraphrases in
the distributional model.  In the simplest case, the sentential context of a
word would simply be the other words in the sentence.  However, consider the
context of the word ``lost'' in the second sentence of \eqref{ex:coref-context}.  

\begin{example}\label{ex:coref-context}
\begin{itemize}
  \item[$p_1$:] In [the final game of the season]$_1$, [the team]$_2$ held on to their lead until overtime
  \item[$p_2$:] But despite that, [they]$_2$ eventually {\bf lost} [it all]$_1$
\end{itemize}
\end{example}

Here we would like to disambiguate ``lost'', but its immediate context, words
like ``despite'' and ``eventually'', gives no indication as to its correct
sense. Our procedure extends the context of the sentence by incorporating all of
the words from all of the phrases that corefer with a word in the immediate
context.  Since coreference chains 1 and 2 have words in $p_2$, the context of
``lost'' ends up including ``final'', ``game'', ``season'', and ``team'' which
give a strong indication of the sense of ``lost''.
Note that using coreference data is stronger than simply expanding the window
because coreferences can cover arbitrarily long distances.

\section{Evaluation}

As a preliminary evaluation of our system, we constructed a set of demonstrative
examples to test our ability to handle the previously discussed phenomena and
their interactions and ran each example with both a theorem prover and Alchemy. 
Note that when running an example in the theorem prover, weights are not
possible, so any rule that would be weighted in an MLN is simply treated as a
``hard clause'' following \citet{bos:emnlp2005}.
% We executed every example discussed in this paper and more.  

\noindent\textbf{Checking the logical form.}
We constructed a list of 72 simple examples  
that exhaustively cover cases of implicativity (positive, negative, null entailments
in both positive and negative environments), hypernymy,
quantification, and the interaction between implicativity and 
hypernymy.  The purpose of these simple tests is to ensure that our flattened
logical form and truth condition rules correctly maintain the semantics of the
underlying DRSs. Examples are given in \eqref{ex:regression}.

\begin{example}\label{ex:regression}
\begin{itemize}
  \item[(a)] The mayor did not manage to build a stadium $\nvDash$ The mayor built a stadium
%  \item[(b)] The mayor did not manage to build a stadium $\vDash$ The mayor did not build a stadium
  \item[(b)] Fido is a dog and every dog walks $\vDash$ A dog walks
%  \item[(d)] Fido is a dog and every dog walks $\nvDash$ No dogs walk
\end{itemize}
\end{example}


\noindent\textbf{Examples in previous sections.}
Examples 
\eqref{ex:imp-fact-nested}, 
\eqref{ex:hope-build}, 
\eqref{ex:prob-wordsense}, 
\eqref{ex:syn-hyp-pos}, and
\eqref{ex:syn-hyp-neg} 
all come out as expected.  Each of these examples demonstrates one of the
phenomena in isolation. However, example (\ref{ex:coref}) returns ``not entailed'',
the incorrect answer.  As discussed previously, this failure is a result of our
system's inability to correctly incorporate the complex coreferring expression
``a replacement for Candlestick Park''.  However, the system {\it is} able to
correctly incorporate the coreference of ``he'' in the second sentence to
``Christopher'' in the first.

\noindent\textbf{Implicativity and word sense.}
For example \eqref{ex:ws-imp-1}, ``fail to'' is a negatively entailing
implicative in a positive environment.
So, $p$ correctly entails $h_{good}$ in both the theorem prover and Alchemy. 
However, the theorem prover incorrectly licenses the entailment of $h_{bad}$
while Alchemy does not.  
The probabilistic approach performs better in this situation because the
categorial approach does not distinguish between a good paraphrase and a
bad one.  This example also demonstrates the advantage of using a
context-sensitive distributional model to calculate the probabilities of
paraphrases because ``reward'' is an {\it a priori} better paraphrase than
``observe'' according to WordNet since it appears in a higher ranked synset. 

\begin{example}\label{ex:ws-imp-1}
\begin{itemize}
  \item[$p$:] The U.S. is watching closely as South Korea fails to honor
  U.S. patents\footnote{Example \eqref{ex:ws-imp-1} is adapted from Penn
  Treebank document wsj\_0020 while \eqref{ex:ws-imp-2} is adapted from document
  wsj\_2358}
  \item[$h_{good}$:] South Korea does not {\bf observe} U.S. patents
  \item[$h_{bad}$:] South Korea does not {\bf reward} U.S. patents
\end{itemize}
\end{example}

\noindent\textbf{Implicativity and hypernymy.}
\citet{maccartney:iwcs2009} extended the work by \citet{nairn:icos2006} in order to
correctly treat inference involving monotonicity and exclusion.  
Our approaches to implicatives and factivity and hyper/hyponymy combine naturally
to address these issues because of the structure of our logical representations
and rules.  For example, no additional work is required to license the
entailments in \eqref{ex:imp-fact-hyper}.  

\begin{example}\label{ex:imp-fact-hyper}
\begin{itemize}
  \item[(a)] John refused to dance $\vDash$ John didn't tango
  \item[(b)] John did not forget to tango $\vDash$ John danced
\end{itemize}
\end{example}

Example \eqref{ex:ws-imp-2} demonstrates how our system combines
categorial implicativity with a probabilistic approach to hypernymy.  
The verb ``anticipate that'' is positively entailing in the negative
environment.
The verb ``moderate'' can mean ``chair'' as in ``chair a discussion'' or
``curb'' as in ``curb spending''.  Since ``restrain'' is a hypernym of
``curb'', it receives a weight based on the applicability of the word ``curb''
in the context.  Similarly, ``talk'' receives a weight based on its
hyponym ``chair''. Since our model predicts ``curb'' to be a more probable
paraphrase of ``moderate'' in this context than ``chair'' (even though the
priors according to WordNet are reversed), the system is able to infer
$h_{good}$ while rejecting $h_{bad}$.

\begin{example}\label{ex:ws-imp-2}
\begin{itemize}
  \item[$p$:] He did not anticipate that inflation would moderate this year
  \item[$h_{good}$:] Inflation {\bf restrained} this year
  \item[$h_{bad}$:] Inflation {\bf talked} this year
\end{itemize} 
\end{example}

% According to \citet{nairn:icos2006}, the verb ``predict that'' is a positively
% entailing factive, but if \eqref{ex:ws-imp-2} is modified so that ``anticipate''
% is replaced by ``predict'', Boxer produces a DRS without a nested proposition
% for ``moderate'', meaning that it is not possible to analyze ``predict that'' as
% a factive and preventing our system from licensing the entailment. 


\noindent\textbf{Word sense, coreference, and hypernymy.}
Example \eqref{ex:ws-coref} demonstrates the interaction between paraphrase,
hypernymy, and coreference incorporated into a single entailment.  The
relevant coreference chains are marked explicitly in the example.  The correct
inference relies on recognizing that ``he'' in the hypothesis refers to ``Joe
Robbie'' and ``it'' to ``coliseum'', which is a hyponym of ``stadium''. 
Further, our model recognizes that ``sizable'' is a better paraphrase for
``healthy'' than ``intelligent'' even though WordNet has the reverse order.

\begin{example}\label{ex:ws-coref}
\begin{itemize}
  \item[$p$:] [Joe Robbie]$_{53}$ couldn't persuade the mayor , so [he]$_{53}$ built [[his]$_{53}$ own coliseum]$_{54}$. 
  \item[] [He]$_{53}$ has used [it]$_{54}$ to turn a healthy profit.\footnote{Only relevent coreferences have been marked}
  \item[$h_{good}$:] {\it Joe Robbie} used {\it a stadium} to turn a {\bf sizable} profit
  \item[$h_{bad-1}$:] {\it Joe Robbie} used {\it a stadium} to turn an {\bf intelligent} profit
  \item[$h_{bad-2}$:] {\it The mayor} used {\it a stadium} to turn a healthy profit
\end{itemize}
\end{example}

\section{Future work}

[pull out realations from the logical form to have more interesting $\alpha$]

[think about vector spaces that take logical form into account similar to how
pado and lapata took dependencies into account.]

[mention vibhav's work]

[mention vectors built form entire sentences.  can measure similarity between
phrases.]



The next step is to execute a full-scale evaluation of our approach using 
more varied phenomena and naturally occurring sentences. 
However, the memory requirements of Alchemy are a limitation that prevents us
from currently executing larger and more complex examples.  The problem arises
because Alchemy considers every possible grounding of every atom, even when a
more focused subset of atoms and inference rules would suffice. There is on-going
work to modify Alchemy so that only the required groundings are incorporated
into the network, reducing the size of the model and thus making it possible to
handle more complex inferences.  We will be able to begin using this new version
of Alchemy very soon and our task will provide an excellent test case for the
modification.

Since Alchemy outputs a probability of entailment, it is necessary to fix a
threshold that separates entailment from nonentailment.
We plan to use machine learning techniques to compute an appropriate threshold
automatically from a calibration dataset such as a corpus of valid and invalid
paraphrases.


\section{Conclusion}

In this paper, we have introduced a system that implements a first step
towards integrating logical semantic representations with
probabilistic weights using methods from Statistical Relational AI,
particularly Markov Logic. We have focused on three phenomena and their
interaction: implicatives, coreference, and word meaning. Taking
implicatives and coreference as categorial and word meaning as
probabilistic, we have used a distributional model to generate
paraphrase appropriateness ratings, which we then transformed into
weights on first order formulas.
The resulting MLN approach is able to correctly solve a number of difficult
textual entailment problems that require handling complex combinations of these
important semantic phenomena.

% The framework we have developed takes a pair of natural
% language sentences as input and parses them into DRS \citep{kamp:book93}
% representations.  It then augments those representations by linking the
% predicates back to the original words in the sentences and incorporating
% coreference information from OntoNotes \citep{hovy:naacl2006}.
% Since DRSs are hierarchical structures, our approach flattens them to a simple
% list of atoms while keeping track of the original structure through DRS labels
% as arguments, allowing inferences to be performed in 
% an MLN. In order to maintain the DRS semantics in the flat logical form, we have 
% hand-written a collection of inference rules that are used in the inference. 
% We also generate a list of inference rules to address the particular
% linguistic phenomena that we are handling.  Categorial rules based
% on implication signatures \citep{nairn:icos2006} are used to handle
% implicativity and factivity.  Weighted rules are used to address issues of
% word meaning in context.



\section*{Acknowledgements}

This work was supported by the Department of Defense (DoD) through a
National Defense Science and Engineering Graduate Fellowship (NDSEG) Fellowship
for the first author, National Science Foundation grant IIS-0845925 for the
second author, and a grant from the Longhorn Innovation Fund for Technology.
[TODO: Is all of this right?]



\bibliographystyle{styles/spbasic}
\bibliography{mln-sem}

\end{document}


\section{Introduction}

Logic-based representations of natural language meaning have a long history.
Representing the meaning of language in a first-order logical form is appealing
because it provides a powerful and flexible way to express even complex
propositions. However, systems built solely using first-order logical forms tend
to be very brittle as they have no way of integrating uncertain knowledge.
They, therefore, tend to have high precision at the cost of low recall
\citep{bos:emnlp2005}.

Recent advances in computational linguistics have yielded robust methods that
use statistically-driven weighted models.  For example, distributional
models of word meaning have been used successfully to judge paraphrase
appropriateness by representing the meaning of a word in context as a point in a
high-dimensional semantics space
\citep{erk:emnlp2008,thater:acl2010,reisinger:naacl2010,dinu:emnlp2010,vandecruys:emnlp2011}.
However, these models only address word meaning, and do not
addres the question of providing meaning representations for complete
sentences. It is a long-standing open question how best to
integrate the weighted or probabilistic information coming from such modules
with logic-based representations in a way that allows for reasoning over both. 
See, for example, \citet{hobbs:alj93}.

The goal of this work is to establish a formal system for combining
logic-based meaning representations with weighted information into a single
unified framework.  This will allow us to obtain the best of both situations: we
will have the full expressivity of first-order logic and be able to reason with
probabilities.  We believe that this will allow for a more complete and robust
approach to natural language understanding.

While this is a large and complex task, this paper proposes first steps toward
our goal by presenting a mechanism for injecting distributional word-similarity
information from a vector space into a first-order logical form.  We
define a mapping from predicate symbols of logical form to points in
vector space, and use it for projecting inference rules from vector
space to logical form. % \citet{gardenfors:book2004} also links logical
% form and a vector space, using the interpretation function, 
% such that logical formulas are interpreted over vector space
% representations. However, the conceptual spaces that he uses are not
% distributional, and points in conceptual space are potential entities.  In contrast, the vector spaces that we use
% are distributional in nature, and, therefore, cannot be interpreted as potential
% entities. % A point in such a space is a potential word, defined through
% % its observed contexts, where the coordinates on each dimension constitute the
% % co-occurrence count with that respective context item.  
% For this reason, we
% define the link between logical form and vector space through a second mapping
% function independent of the interpretation function, which we call the
% \emph{lexical mapping} function.
A central property of distributional vector space models is that they
can predict semantic similarity based on proximity in space.  This comes from
the hypothesis that a word's meaning is its use, and similarity in
meaning is reflected in similarity in use~\citep{Wittgenstein,firth:slaj1957,harris:wj1954}. When syntax is suitably restrained, vector space models can
also be used to predict substitutability in context
\citep{lin:nlej2001,mitchell:acl2008,erk:emnlp2008,thater:acl2010,reisinger:naacl2010,dinu:emnlp2010,vandecruys:emnlp2011}.
So if $A$ and $B$ are words or
expressions that are distributionally similar and that can occur in
the same syntactic contexts, we can substitute $B$ for $A$ in all
contexts that are suitable to both words. This can be described
through the inference rule $A \to B$, with a weight computed from the
vector space. Thus, our main aim in linking logical form to a vector
space in this paper is to project inferences from the vector space to
logical form.

In this paper, we first present our formal framework for projecting inferences
from vector space to logical form.  We then show how that framework can be
applied to a real logical language and vector space to address issues of
ambiguity in word meaning.  Finally, we show how the weighted inference rules
produced by our approach interact appropriately with the first-order logical
form to produce correct inferences.

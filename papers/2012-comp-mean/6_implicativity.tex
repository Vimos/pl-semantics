\section*{Implicatives and factives}

[TODO: It would be nice if this also fit into the interface framework\ldots]

\citet{nairn:icos2006} presented an approach to the treatment of inferences
involving implicatives and factives.  Their approach identifies an ``implication
signature'' for every implicative or factive verb that determines the truth
conditions for the verb's nested proposition, whether in a positive or negative
environment.  Implication signatures take the form ``$x/y$'' where $x$
represents the implicativity in the the positive environment and $y$ represents
the implicativity in the negative environment.  Both $x$ and $y$ have three
possible values: ``+'' for positive entailment, meaning the nested proposition
is entailed, ``-'' for negative entailment, meaning the negation of the proposition
is entailed, and ``o'' for ``null'' entailment, meaning that neither the
proposition nor its negation is entailed. Figure \ref{fig:imp-sig} gives
concrete examples.% for two signatures.

% For example, the verb ``managed to'' has positive entailment in the {\it true}
% case and negative entailment under negation.  So, {\it he managed to escape
% $\vDash$ he escaped} while {\it he did not manage to escape $\vDash$ he did not
% escape}.  On the other hand, the verb ``refused to'' has negative entailment
% in the positive case and ``null'' entailment under negation.  So, {\it he
% refused to fight $\vDash$ he did not fight} but {\it he did not refuse to fight}
% entails neither {\it he fought} nor {\it he did not fight}.

\begin{figure}
\begin{center}
  \begin{tabular}{l c l}
    \hline
   	 & ~~~~~~signature~~~~~~ &  \multicolumn{1}{c}{example} \\
   	\hline
%    	admitted that    & +/+ & he admitted that he knew $\vDash$ he knew \\
%    	                 &     & he did not admit that he knew $\vDash$ he knew \\
%    	\hline
   	forgot that      & +/+ & he forgot that Dave left $\vDash$ Dave left \\
   	                 &     & he did not forget that Dave left $\vDash$ Dave left \\
   	\hline
%    	pretended that   & -/- & he pretended he was sick $\vDash$ he was not sick \\
%    	                 &     & he did not pretend he was sick $\vDash$ he was not sick\\
%    	\hline
%    	wanted to        & o/o & he wanted to fly $?$ he flew \\  
%    	                 &     & he did not want to fly $?$ he flew \\
%    	\hline
   	managed to       & +/- & he managed to escape $\vDash$ he escaped \\
   	                 &     & he did not manage to escape $\vDash$ he did not escape \\
   	\hline
%    	was forced to    & +/o & he was forced to sell $\vDash$ he sold \\
%    	                 &     & he was not forced to sell $?$ he sold \\
%    	\hline
%    	was permitted to & o/- & he was permitted to leave $?$ he left \\
%    	                 &     & he was not permitted to leave $\vDash$ he did not leave \\
%    	\hline
   	forgot to        & -/+ & he forgot to pay $\vDash$ he did not pay \\
   	                 &     & he did not forget to pay $\vDash$ he paid \\
   	\hline
   	refused to       & -/o & he refused to fight $\vDash$ he did not fight \\
   	                 &     & he did not refuse to fight $\nvDash$ \{he fought, he did not fight\} \\
   	\hline
%    	hesitated to     & o/+ & he hesitated to ask $?$ he asked\\
%    	                 &     & he did not hesitate to ask $\vDash$ he asked \\
%    	\hline
  \end{tabular}
\end{center}
\caption{Implication Signatures}
\label{fig:imp-sig}
\end{figure}


\subsection*{Inferences with nested propositions}

The standard conversion from DRT to first-order logic (FOL) (the one used by
Boxer), falls short on nested propositions.  Consider the entailment pair ``John
did not manage to leave'' and ``John left''.  The DRT interpretation and its
corresponding FOL conversion are are shown in Figure \ref{drs:impl-1}.

\begin{figure}
  \centering
  \subfloat[DRT interpretation]{\label{drs:impl-1-drt}
    ~~~~~~~~
	\drs{~x0~}{
	  ~john_{1001}(x0)~ \\
	  \negdrs{~e1 p2~}{
	    ~manage_{1004}(e1)~ \\
	    ~agent(e1, x0)~ \\
	    ~theme(e1, p2)~ \\
	    ~p2:~\drs{~e3~}{
	      ~leave_{1006}(e3)~ \\
	      ~agent(e3, x0)~
	    }
	  }
	}
	~~~~~~~~
  }
  ~~~~~~~~~
  \subfloat[FOL translation]{\label{drs:impl-1-fol}
    \dhgboxed{$
      ~\exists~ x0.[
      ~  john_{1001}(x0) ~\&~ \\
      ~  \hspace{18px} \lnot \exists~ e1 p2.[
      ~                  manage_{1004}(e1) ~\&~ \\
      ~    \hspace{55px} agent(e1, x0) ~\&~ \\
      ~    \hspace{55px} theme(e1, p2) ~\&~ \\
      ~    \hspace{55px} \exists~ e3.[
      ~                    leave_{1006}(e3) ~\&~ \\
      ~      \hspace{77px} agent(e3, x0) ]]]
    $}
%     ~~~~~~~~
% 	\drs{~x0 e1~}{
% 	  ~john_{2001}(x0)~ \\
% 	  ~leave_{2002}(e1)~ \\
% 	  ~agent(e1, x0)~
% 	}
% 	~~~~~~~~
  }
  \caption{Boxer's DRT interpretation of ``John did not manage to leave.''.}
  \label{drs:impl-1}
\end{figure}

While it should be clear that ``John did not manage to leave'' does {\it not}
entail ``John left'' (and, in fact, entails the opposite), the FOL formula shown
in Figure \ref{drs:impl-1-fol} {\it does} entail the FOL representation of
``John left'' \[ \exists~ x0~e1.[john_{1001}(x0) ~\&~ leave_{1006}(e1) ~\&~
agent(e1, x0)] \]  

The incorrect inference occurs here because the standard DRT-to-FOL translation
loses some information.  DRT expressions are allowed to have {\it labeled
subexpressions}, such as $p2$ in Figure \ref{drs:impl-1-drt} that is used to
reference the {\theme} of the {\it manage} event: the {\it leave} event.  The
FOL expression, on the other hand, shows that $p2$ is the theme of event $e1$,
but has no way of stating what $p2$ refers to.

In order to capture the information that the DRT labels provide, we modify the
DRT expression to contain explicit {\it subexpression triggers}.  That is, for a
sub-DRS $A$ labeled by $p$, we replace $A$ with two new expressions in the same
scope: $POS(p) \to A$ and $NEG(p) \to \lnot A$.  The result of such a
replacement on the DRS in \ref{drs:impl-1-drt} can be see in Figure
\ref{drs:impl-2-drt}.   

\begin{figure}
  \centering
  \subfloat[DRT interpretation with subexpression triggers]{\label{drs:impl-2-drt}
	\drs{~x0~}{
	  ~john_{1001}(x0)~ \\
	  ~\drs{~p2~}{
	    ~\negdrs{~e1~}{
	      ~manage_{1004}(e1)~ \\
	      ~theme(e1, p2)~ \\
	      ~agent(e1, x0)~} \\
	    \unboxedifdrs{POS(p2)}{   \drs{~e3~}{~leave_{1006}(e3)~ \\ ~agent(e3, x0)~}} \\
	    \unboxedifdrs{NEG(p2)}{\negdrs{~e3~}{~leave_{1006}(e3)~ \\ ~agent(e3, x0)~}}
	  }
	}
  }
  ~~~~~~~~~
  \subfloat[Subexpression-triggering inference rules for implicative ``manage
            to'' with signature +/-]{\label{drs:impl-2-rules} 
  \shortstack{
      \unboxedifdrs{\drs{~p~}{~   \drs{~e~}{~manage_{1004}(e)~ \\ ~theme(e, p)~}~}}{POS(p)} \\
      \\\\\\\\\\\\
      \unboxedifdrs{\drs{~p~}{~\negdrs{~e~}{~manage_{1004}(e)~ \\ ~theme(e, p)~}~}}{NEG(p)}
    }
  }
  \caption{First (insufficient) attempt at correcting for the loss of labeled
  sub-expression information.}
  \label{drs:impl-2}
\end{figure}

Now that our labeled subexpression has triggers, we can introduce inference
rules to activate those triggers.  The purpose of these inference rules is to
capture the behavior dictated by the implication signature of the implicative
verb for which the relevant subexpression is the theme.  For example, according
to the implication signature, the implicative {\it manage to} is positively
entailing in positive contexts and negatively entailing in negative contexts.
This means that if John {\it managed to} do what is described by $p$, then the
event described by $p$ occurred, or in other words, the subexpression of $p$ is
{\it true}. Likewise, if John {\it did not manage to} do what is described by
$p$, then the event described by $p$ {\it did not} occur, meaning that the
subexpression of $p$ is {\it false}.  

The triggering inference rules for {\it managed to} are shown in Figure
\ref{drs:impl-2-rules}.  The first rule, for positive contexts, says that for
all propositions $p$, if $p$ is ``managed'', then $p$'s subexpression is {\it
true}, so trigger the ``positive entailment'' subexpression which, in our
example, says that the {\it leaving} event occurred.  The second rule, for
negative contexts, says that for all propositions $p$, if there is {\it no}
``managing'' of $p$, then $p$'s subexpression is {\it false}, so trigger the
``negative entailment'' subexpression to say that there is {\it no} event of
{\it leaving}.

While this approach works for positive contexts, there is a subtle problem for
negative contexts.  The negative context rule in Figure \ref{drs:impl-2-rules}
can be translated to FOL as \[ \forall~ p.[~ \lnot \exists~ e.[~ manage(e) \land
theme(e,p)) ~] \to NEG(p) ~] \] This expression is stating that for all
propositions $p$, $p$ is {\it false} if there is no ``managing'' of $p$.  Now,
we want this inference rule to be used in cases where it is stated that
``managing'' did not occur, such as in the expression of Figure
\ref{drs:impl-2-drt}, where we see that is is the case that \[ ~\negdrs{~x0 e1
p2~}{ ~manage_{1004}(e1)~ \\ ~agent(e1, x0)~ \\ ~theme(e1, p2)~} \] which is
equivalent to the FOL expression \[ \lnot manage_{1004}(e1) \land \lnot
agent(e1, x0) \land \lnot theme(e1, p2) \] stating that $e1$ is a ``managing''
of $p2$ by $x0$.  However, the antecedent of our negative context rule states
that there is {\it no} ``managing'' of the proposition, so the rule would only
be used if it could be proven that there is no ``managing'' at all.
Unfortunately, stating that $p2$ is not ``managed'' {\it by x0} does {\it not}
entail that $p2$ is not ``managed'' at all since $p2$ could be managed by
someone other than $x0$.

To overcome this problem, we modify our representation of a negated event. 
Instead of representing an event, such as the ``managing'' event, that that
{\it did not} occur as $\lnot \exists~ e.[~ manage(e) ~]$, we represent it
explicitly as an event of {\it non-occurrence}: $\exists~ e.[~
\textbf{not\_}manage(e) ~]$.  Applying this change to the DRS and inference
rules in Figure \ref{drs:impl-2}, we arrive at our final form in Figure
\ref{drs:impl-3}.

\begin{figure}
  \centering
  \subfloat[DRT interpretation with subexpression triggers]{\label{drs:impl-3-drt}
	\drs{~x0~}{
	  ~john_{1001}(x0)~ \\
	  ~\drs{~e1 p2~}{
	    ~\textbf{not\_}manage_{1004}(e1)~ \\
	    ~theme(e1, p2)~ \\
	    \negdrs{ }{~agent(e1, x0)~} \\
	    \unboxedifdrs{POS(p2)}{\drs{~e3~}{~              leave_{1006}(e3)~ \\ ~agent(e3, x0)~}} \\
	    \unboxedifdrs{NEG(p2)}{\drs{~e3~}{~\textbf{not\_}leave_{1006}(e3)~ \\ \negdrs{ }{~agent(e3, x0)~}}}
	  }
	}
  }
  ~~~~~~~~~
  \subfloat[Subexpression-triggering inference rules for implicative ``manage
            to'' with signature +/-]{\label{drs:impl-3-rules} 
  \shortstack{
      \unboxedifdrs{\drs{~p~}{~\drs{~e~}{~              manage_{1004}(e)~ \\ ~theme(e, p)~}~}}{POS(p)} \\
      \\\\\\\\\\\\
      \unboxedifdrs{\drs{~p~}{~\drs{~e~}{~\textbf{not\_}manage_{1004}(e)~ \\ ~theme(e, p)~}~}}{NEG(p)}
    }
  }
  \caption{Explicit capturing of sub-expression information.}
  \label{drs:impl-3}
\end{figure}

Using this strategy, we can see that the negative context rule is active when
there exists a ``not-managing'' event, and the representation of ``John did
not manage to leave'' explicitly states that there is such an event, meaning
that the rule will be used in the inference.  With all of these pieces in place,
the inference works as expected.

Thus, we transform the output of Boxer in two ways.  First, we identify any
labeled propositions and replace them with pairs of proposition triggers.  Then,
we modify any negated DRSs by extracting the verb and theme atoms, changing the
verb predicate to a ``not\_'' predicate, and finally ensuring that all other
expressions under the negated DRS (aside from the labeled proposition itself),
remain under a negated DRS.

Once the sentence representations have been modified, we generate inference
rules for each implicative verb.  If the verb is positively entailing in
positive contexts, we generate a rule of the form 
\[ \forall~ p.[~ \exists~ e.[~ \langle verb \rangle(e) \land theme(e,p)) ~] \to POS(p) ~] \]
but if it is negatively entailing in positive contexts, we instead generate a rule of the form
\[ \forall~ p.[~ \exists~ e.[~ \langle verb \rangle(e) \land theme(e,p)) ~] \to NEG(p) ~] \]
If the verb is positively entailing in {\it negative} contexts, we generate a rule of the form
\[ \forall~ p.[~ \exists~ e.[~ \textbf{not\_}\langle verb \rangle(e) \land theme(e,p)) ~] \to POS(p) ~] \]
but if it is negatively entailing in negative contexts, we instead generate a rule of the form
\[ \forall~ p.[~ \exists~ e.[~ \textbf{not\_}\langle verb \rangle(e) \land theme(e,p)) ~] \to NEG(p) ~] \]
If the verb is non-entailing in either positive or negative contexts, then we do
not generate a rule for that context polarity. 

This approach works for arbitrarily long chains of nested implicatives.  It also
interacts properly with 



\subsection*{Use of subcategorization frame}

The the verb {\it forget} has different implicative properties depending on its
subcategorization frame.  When used as {\it forget that}, as in ``He forgot that
Dave left'', it is positively entailing in both positive and negative contexts. 
When used as {\it forget to}, as in ``He forgot to lock the door'', it is
negatively entailing in positive contexts and positively entailing in negative
contexts.

In order to generate the correct inference rules, the right implication
signature must be selected.  To do this, we make use of the dependency parse
generated by the C\&C parser that is input to Boxer.  The parse tells us which
version of the verb is being used.  In our example ``He forgot that
Dave left'', {\it forgot} has a ``ccomp'' relationship to the {\it left}, while
in ``He forgot to lock the door'', {\it forgot} has a ``xcomp+to'' relationship
to {\it lock}.



\subsection*{Weighting implicative inference rules}

While we do not have a scheme for generate such weights at this time, one of the
advantages of our approach is that it allows for the independent weighting of
each implicative inference rule in isolation.  

